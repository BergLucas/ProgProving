\documentclass{article}

% Packages
\usepackage[a4paper]{geometry}
\usepackage[utf8]{inputenc}
\usepackage{amsmath}
\usepackage{progproving}
\usepackage{listings}

\geometry{top=25mm,bottom=25mm,inner=25mm,outer=20mm}
\allowdisplaybreaks

\begin{document}
\title{ProgProving package : Example}

\author{BERG Lucas}

\maketitle

\section{Introduction}
This document is an example for the ProgProving package. We will prove the following function:

\begin{lstlisting}[language=C]
int sum(int array[], int length) {
    int total = 0;
    int i = 0;
    while (i < length) {
        total = total + array[i];
        i = i + 1;
    }
    return total;
}
\end{lstlisting}

\section{Specifications}

First, let's write the specifications of the function.

\subsection{Header}
\begin{lstlisting}[language=C]
int sum(int array[], int length)
\end{lstlisting}

\subsection{Environment}
/

\subsection{Precondition}
\begin{flalign*}
  & \pre \equiv \state{
    & 0 \leq length \land \\
    & length = \text{size of array[]}
  } &
\end{flalign*}

\subsection{Postcondition}
\begin{flalign*}
  & \post \equiv \state{
    & 0 \leq length \land \\
    & length = \text{size of array[]} \land \\
    & sum = \sum^{length-1}_{j=0} array[j] \land \\
    & length = length_0 \land \\
    & array = array_0
  } &
\end{flalign*}

with:
\begin{itemize}
  \item $length_0$ which is the value of $length$ before the function call
  \item $array_0$ which is the value of $array$ before the function call
\end{itemize}

\section{Proof using the strongest postcondition (sp)}

\subsection{Definitions}
\begin{flalign*}
  & \init \equiv
  \seq{
    & total := 0; \\
    & i := 0;
  }
  & \\
  & \iter \equiv
  \seq{
    & total := total + array[i]; \\
    & i := i + 1;
  }
  & \\
  & \term \equiv  sum := total;
\end{flalign*}

\subsection{The instructions before the loop}

First, we need to calculate the instructions before the loop by doing $\state{\pre} \init \state{\lpre}$.

\begin{flalign*}
  & \ppsp{\init}{\state{\pre}} & \\
  %
  \equiv & \ppsp{
    \seq{
      & total := 0; \\
      & i := 0;
    }
  }{
    \state{
      & 0 \leq length \land \\
      & length = \text{size of array[]}
    }
  } & \\
  %
  \equiv & \ppsp{
    i := 0;
  }{
    \state{
      & 0 \leq length \land \\
      & length = \text{size of array[]}
    }[total/total']
    \land
    \state{
      total := 0
    }[total/total']
  } & \\
  %
  \equiv & \ppsp{
    i := 0;
  }{
    \state{
      & 0 \leq length \land \\
      & length = \text{size of array[]}
    }
    \land
    \state{
      total = 0
    }
  } & \\
  %
  \equiv & \ppsp{
    i := 0;
  }{
    \state{
      & 0 \leq length \land \\
      & length = \text{size of array[]} \land \\
      & total = 0
    }
  } & \\
  %
  \equiv & \state{
    & 0 \leq length \land \\
    & length = \text{size of array[]} \land \\
    & total = 0
  }[i/i']
  \land
  \state{
    i := 0
  }[i/i'] & \\
  %
  \equiv & \state{
    & 0 \leq length \land \\
    & length = \text{size of array[]} \land \\
    & total = 0
  }
  \land
  \state{
    i = 0
  } & \\
  %
  \equiv & \state{
    & 0 \leq length \land \\
    & length = \text{size of array[]} \land \\
    & total = 0 \land \\
    & i = 0
  } & \\
  %
  \equiv & \lpre &
\end{flalign*}

All right, we got the loop precondition, so we can move on to the next step.

\subsection{The loop instruction}

Now, we need to prove the loop instruction. There are 4 steps.

\subsubsection{Find the loop invariant $\inv$}
First, we need to find the loop invariant $\inv$. In most cases, the invariant will describe the state of the loop using i-1. In our case, the invariant is:

\begin{flalign*}
  & \inv \equiv \state{
    & 0 \leq i \leq length \land \\
    & length = \text{size of array[]} \land \\
    & total = \sum^{i-1}_{j = 0} array[j]
  } &
\end{flalign*}

\subsubsection{Proof that $\inv$ is an invariant}
Now, we have to prove $\{ \inv \land \cond \} \iter \{ \inv \}$. If we can prove that, this means that the invariant is correct.
First, let's define $\inv \land \cond$

\begin{flalign*}
  & \inv \land \cond \equiv \state{
    & 0 \leq i \leq length \land \\
    & length = \text{size of array[]} \land \\
    & total = \sum^{i-1}_{j = 0} array[j]
  } \land \state{
    i < length
  } \equiv \state{
    & 0 \leq i < length \land \\
    & length = \text{size of array[]} \land \\
    & total = \sum^{i-1}_{j = 0} array[j]
  } &
\end{flalign*}

Then, we can do a sp to check if $\inv$ is an invariant.

\begin{flalign*}
  & \ppsp{\iter}{\inv \land \cond} & \\
  %
  \equiv & \ppsp{
    \seq{
      & total := total + array[i]; \\
      & i := i + 1;
    }
  }{
    \state{
      & 0 \leq i < length \land \\
      & length = \text{size of array[]} \land \\
      & total = \sum^{i-1}_{j = 0} array[j]
    }
  } & \\
  %
  \equiv & \ppsp{
    i := i + 1;
  }{
    \state{
      & 0 \leq i < length \land \\
      & length = \text{size of array[]} \land \\
      & total = \sum^{i-1}_{j = 0} array[j]
    }[total/total']
    \land
    \state{
      total := total + array[i]
    }[total/total']
  } & \\
  %
  \equiv & \ppsp{
    i := i + 1;
  }{
    \state{
      & 0 \leq i < length \land \\
      & length = \text{size of array[]} \land \\
      & total' = \sum^{i-1}_{j = 0} array[j]
    }
    \land
    \state{
      total = total' + array[i]
    }
  } & \\
  %
  \equiv & \ppsp{
    i := i + 1;
  }{
    \state{
      & 0 \leq i < length \land \\
      & length = \text{size of array[]} \land \\
      & total' = \sum^{i-1}_{j = 0} array[j]
    }
    \land
    \state{
      total - array[i] = total'
    }
  } & \\
  %
  \equiv & \ppsp{
    i := i + 1;
  }{
    \state{
      & 0 \leq i < length \land \\
      & length = \text{size of array[]} \land \\
      & total - array[i] = \sum^{i-1}_{j = 0} array[j]
    }
  } & \\
  %
  \equiv & \ppsp{
    i := i + 1;
  }{
    \state{
      & 0 \leq i < length \land \\
      & length = \text{size of array[]} \land \\
      & total = (\sum^{i-1}_{j = 0} array[j]) + array[i]
    }
  } & \\
  %
  \equiv & \ppsp{
    i := i + 1;
  }{
    \state{
      & 0 \leq i < length \land \\
      & length = \text{size of array[]} \land \\
      & total = \sum^{i}_{j = 0} array[j]
    }
  } & \\
  %
  \equiv & \state{
    & 0 \leq i < length \land \\
    & length = \text{size of array[]} \land \\
    & total = \sum^{i}_{j = 0} array[j]
  }[i/i'] \land \state{
    i := i + 1
  }[i/i'] & \\
  %
  \equiv & \state{
    & 0 \leq i' < length \land \\
    & length = \text{size of array[]} \land \\
    & total = \sum^{i'}_{j = 0} array[j]
  } \land \state{
    i = i' + 1
  } & \\
  %
  \equiv & \state{
    & 0 \leq i' < length \land \\
    & length = \text{size of array[]} \land \\
    & total = \sum^{i'}_{j = 0} array[j]
  } \land \state{
    i - 1 = i'
  } & \\
  %
  \equiv & \state{
    & 0 \leq i-1 < length \land \\
    & length = \text{size of array[]} \land \\
    & total = \sum^{i-1}_{j = 0} array[j]
  } & \\
  %
  \equiv & \state{
    & 0 < i \leq length \land \\
    & length = \text{size of array[]} \land \\
    & total = \sum^{i-1}_{j = 0} array[j]
  } & \\
  %
  & \Rightarrow \inv &
\end{flalign*}

All right, the invariant is correct, so we can move on to the next step.

\subsubsection{Proof that $\inv$ is strong enough}
The next step is to prove that the invariant is true before the loop, so we need to prove $\lpre \Rightarrow \inv$.

\begin{flalign*}
  \lpre \equiv & \state{
    & 0 \leq length \land \\
    & length = \text{size of array[]} \land \\
    & total = 0 \land \\
    & i = 0
  } & \\
  %
  \equiv & \state{
    & 0 = i \land i \leq length \land \\
    & length = \text{size of array[]} \land \\
    & total = \sum^{i-1}_{j = 0} array[j]
  } & \\
  %
  & \Rightarrow \inv
\end{flalign*}

All right, the invariant is strong enough, so we can move on to the next step.

\subsubsection{Calculate the strongest loop postcondition $\lpost$}
The last step is to calculate the strongest loop postcondition $\lpost$, so we need to simplify $\state{\inv \land \neg \cond}$.

\begin{flalign*}
  \inv \land \neg \cond \equiv & \state{
    & 0 \leq i \leq length \land \\
    & length = \text{size of array[]} \land \\
    & total = \sum^{i-1}_{j = 0} array[j]
  } \land \state{
    i \geq length
  } & \\
  \equiv & \state{
    & 0 \leq i \land i = length \land \\
    & length = \text{size of array[]} \land \\
    & total = \sum^{i-1}_{j = 0} array[j]
  } & \\
  \equiv & \state{
    & 0 \leq length \land \\
    & length = \text{size of array[]} \land \\
    & total = \sum^{length-1}_{j = 0} array[j]
  } & \\
  \equiv & \lpost
\end{flalign*}

\subsection{The instructions after the loop}
Finally, we need to calculate the instructions after the loop by doing $\state{\lpost} \term \state{\post}$.

\begin{flalign*}
  & \ppsp{\term}{\lpost} & \\
  %
  \equiv & \ppsp{
    sum := total;
  }{
    \state{
      & 0 \leq length \land \\
      & length = \text{size of array[]} \land \\
      & total = \sum^{length-1}_{j = 0} array[j]
    }
  } & \\
  %
  \equiv & \state{
    & 0 \leq length \land \\
    & length = \text{size of array[]} \land \\
    & total = \sum^{length-1}_{j = 0} array[j]
  }[sum/sum']
  \land
  \state{
    sum := total;
  }[sum/sum'] & \\
  %
  \equiv & \state{
    & 0 \leq length \land \\
    & length = \text{size of array[]} \land \\
    & total = \sum^{length-1}_{j = 0} array[j]
  } \land \state{
    sum = total
  } & \\
  %
  \equiv & \state{
    & 0 \leq length \land \\
    & length = \text{size of array[]} \land \\
    & sum = \sum^{length-1}_{j = 0} array[j]
  } & \\
  %
  & \Rightarrow \post &
\end{flalign*}

Everything is proved, so the program is correct.
\end{document}