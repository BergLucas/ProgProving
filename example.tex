\documentclass{article}

% Packages
\usepackage[a4paper]{geometry}
\usepackage[utf8]{inputenc}
\usepackage{amsmath}
\usepackage{progproving}
\usepackage{listings}

\geometry{top=25mm,bottom=25mm,inner=25mm,outer=20mm}
\allowdisplaybreaks

\begin{document}
\title{ProgProving package : Example}

\author{BERG Lucas}

\maketitle

\section{Introduction}
    This document is an example for the ProgProving package. We will prove the following function :

    \begin{lstlisting}[language=C]
int sum(int array[], int length) {
    int total = 0;
    int i = 0;
    while (i < length) {
        total = total + array[i];
        i = i + 1;
    }
    return total;
}
    \end{lstlisting}

\section{Specifications}
    First, let's write the specifications of the function.
    
    \subsection{Header}
        \begin{lstlisting}[language=C]
int sum(int array[], int length)
        \end{lstlisting}
    
    \subsection{Environment}
        /
    
    \subsection{Preconditions}
        \begin{flalign*}
            & \pre = \state{
                0 \leq length = \text{size of array[]}
            } &
        \end{flalign*}
    
    \subsection{Postconditions}
        \begin{flalign*}
            & \post = \state{
                & length = length_0 \wedge \\
                & array = array_0 \wedge \\
                & sum = \sum^{length-1}_{j=0} array[j]
            } &
        \end{flalign*}

\section{Proof using the strongest postcondition (sp)}
    \subsection{Definitions}
        \begin{flalign*}
            & \init \equiv
                \begin{aligned}
                    & total := 0; \\
                    & i := 0;
                \end{aligned}
            & \\
            & \iter \equiv 
                \begin{aligned}
                    & total := total + array[i]; \\
                    & i := i + 1;
                \end{aligned}
            & \\
            & \term \equiv  sum := total; & \\
            & \text{Loop condition} : \cond \equiv i < length & \\
            & \text{Loop preconditions} : \lpre = \state{
                & 0 \leq length = \text{size of array[]} \wedge \\
                & total = 0 \wedge \\
                & i = 0
            } &
        \end{flalign*}
    
    \subsection{Find the loop invariant}
        First, we need to find the loop invariant. In most cases, the invariant will describe the state of the loop using i-1. In our case, the invariant is :
        \begin{flalign*}
            & \inv = \state{
                & 0 \leq i \leq length \wedge \\
                & total = \sum^{i-1}_{j = 0} array[j] \wedge \\
                & length = length_0 \wedge \\
                & array = array_0
            } &
        \end{flalign*}
    
    \subsection{Proof that $\inv$ is an invariant}
        Now, we have to prove $\{ \inv \wedge \cond \} \iter \{ \inv \}$. If we can prove that, this means that the invariant is correct.
        First, let's define $\inv \wedge \cond$
        \begin{flalign*}
            & \inv \wedge \cond = \state{
                & 0 \leq i \leq length \wedge \\
                & total = \sum^{i-1}_{j = 0} array[j] \wedge \\
                & length = length_0 \wedge \\
                & array = array_0
            } \wedge \state{
                i < length
            } = \state{
                & 0 \leq i < length \wedge \\
                & total = \sum^{i-1}_{j = 0} array[j] \wedge \\
                & length = length_0 \wedge \\
                & array = array_0
            } &
        \end{flalign*}

        Then, we can do a sp to check if $\inv$ is an invariant.

        \begin{flalign*}
                & \ppsp{\iter}{\inv \wedge \cond} & \\
            %
            %
            %
            =   & \ppsp{
                    \begin{aligned}
                        & total := total + array[i]; \\
                        & i := i + 1;
                    \end{aligned}
                }{
                    \state{
                        & 0 \leq i < length \wedge \\
                        & total = \sum^{i-1}_{j = 0} array[j] \wedge \\
                        & length = length_0 \wedge \\
                        & array = array_0
                    }
                } & \\
            %
            %
            %
            =   & \ppsp{
                    i := i + 1;
                }{
                    \state{
                        & 0 \leq i < length \wedge \\
                        & total = \sum^{i-1}_{j = 0} array[j] \wedge \\
                        & length = length_0 \wedge \\
                        & array = array_0
                    }[total/total']
                    \wedge
                    \state{
                        total := total + array[i]
                    }[total/total']
                } & \\
            %
            %
            %
            =   & \ppsp{
                    i := i + 1;
                }{
                    \state{
                        & 0 \leq i < length \wedge \\
                        & total' = \sum^{i-1}_{j = 0} array[j] \wedge \\
                        & length = length_0 \wedge \\
                        & array = array_0
                    }
                    \wedge
                    \state{
                        total = total' + array[i]
                    }
                } & \\
            %
            %
            %
            =   & \ppsp{
                    i := i + 1;
                }{
                    \state{
                        & 0 \leq i < length \wedge \\
                        & total' = \sum^{i-1}_{j = 0} array[j] \wedge \\
                        & length = length_0 \wedge \\
                        & array = array_0
                    }
                    \wedge
                    \state{
                        total - array[i] = total'
                    }
                } & \\
            %
            %
            %
            =   & \ppsp{
                    i := i + 1;
                }{
                    \state{
                        & 0 \leq i < length \wedge \\
                        & total - array[i] = \sum^{i-1}_{j = 0} array[j] \wedge \\
                        & length = length_0 \wedge \\
                        & array = array_0
                    }
                } & \\
            %
            %
            %
            =   & \ppsp{
                    i := i + 1;
                }{
                    \state{
                        & 0 \leq i < length \wedge \\
                        & total = (\sum^{i-1}_{j = 0} array[j]) + array[i] \wedge \\
                        & length = length_0 \wedge \\
                        & array = array_0
                    }
                } & \\
            %
            %
            %
            =   & \ppsp{
                    i := i + 1;
                }{
                    \state{
                        & 0 \leq i < length \wedge \\
                        & total = \sum^{i}_{j = 0} array[j] \wedge \\
                        & length = length_0 \wedge \\
                        & array = array_0
                    }
                } & \\
            %
            %
            %
            =   & \state{
                    & 0 \leq i < length \wedge \\
                    & total = \sum^{i}_{j = 0} array[j] \wedge \\
                    & length = length_0 \wedge \\
                    & array = array_0
                }[i/i'] \wedge \state{
                    i := i + 1
                }[i/i'] & \\
            %
            %
            %
            =   & \state{
                    & 0 \leq i' < length \wedge \\
                    & total = \sum^{i'}_{j = 0} array[j] \wedge \\
                    & length = length_0 \wedge \\
                    & array = array_0
                } \wedge \state{
                    i = i' + 1
                } & \\
            %
            %
            %
            =   & \state{
                    & 0 \leq i' < length \wedge \\
                    & total = \sum^{i'}_{j = 0} array[j] \wedge \\
                    & length = length_0 \wedge \\
                    & array = array_0
                } \wedge \state{
                    i - 1 = i'
                } & \\
            %
            %
            %
            =   & \state{
                    & 0 \leq i-1 < length \wedge \\
                    & total = \sum^{i-1}_{j = 0} array[j] \wedge \\
                    & length = length_0 \wedge \\
                    & array = array_0
                } & \\
            %
            %
            %
            =   & \state{
                    & 0 < i \leq length \wedge \\
                    & total = \sum^{i-1}_{j = 0} array[j] \wedge \\
                    & length = length_0 \wedge \\
                    & array = array_0
                } & \\
            %
            %
            %
            & \Rightarrow \inv &
        \end{flalign*}

        All right, the invariant is correct so we can move on to the next step.
    
    \subsection{Proof that $\inv$ is strong enough}
        The next step is to prove that the invariant is true before the loop so we need to prove $\lpre \Rightarrow \inv$.

        \begin{flalign*}
            \lpre = & \state{
                        & 0 \leq length = \text{size of array[]} \wedge \\
                        & total = 0 \wedge \\
                        & i = 0
                    } & \\
            %
            %
            %
                  = & \state{
                        & 0 = i \leq length = \text{size of array[]} \wedge \\
                        & total = \sum^{-1}_{j = 0} array[j] = 0
                    } & \\
            %
            %
            %
                    & \Rightarrow \inv
        \end{flalign*}

        All right, the invariant is strong enough so we can move on to the next step.

    \subsection{Proof that the program will return the postconditions after the loop}
        The next step is to prove that the program will return the postconditions after the loop so we need to prove $\{ \inv \wedge \neg \cond \} \term \{ \post \}$.
        First, let's define $\inv \wedge \neg \cond$
        \begin{flalign*}
            \inv \wedge \neg \cond & = \state{
                & 0 \leq i \leq length \wedge \\
                & total = \sum^{i-1}_{j = 0} array[j] \wedge \\
                & length = length_0 \wedge \\
                & array = array_0
            } \wedge \state{
                i \geq length
            } = \state{
                & i = length \wedge \\
                & total = \sum^{i-1}_{j = 0} array[j] \wedge \\
                & length = length_0 \wedge \\
                & array = array_0
            } & \\
            & = \state{
                & total = \sum^{length-1}_{j = 0} array[j] \wedge \\
                & length = length_0 \wedge \\
                & array = array_0
            } &
        \end{flalign*}

        Then, we can do a sp.

        \begin{flalign*}
                & \ppsp{\term}{\inv \wedge \neg \cond} & \\
            %
            %
            %
            =   & \ppsp{
                    sum := total;
                }{
                    \state{
                        & total = \sum^{length-1}_{j = 0} array[j] \wedge \\
                        & length = length_0 \wedge \\
                        & array = array_0
                    }
                } & \\
            %
            %
            %
            =   & \state{
                    & total = \sum^{length-1}_{j = 0} array[j] \wedge \\
                    & length = length_0 \wedge \\
                    & array = array_0
                } \wedge \state{
                    sum := total;
                } & \\
            %
            %
            %
            =   & \state{
                    & total = \sum^{length-1}_{j = 0} array[j] \wedge \\
                    & length = length_0 \wedge \\
                    & array = array_0
                } \wedge \state{
                    sum = total;
                } & \\
            %
            %
            %
            =   & \state{
                    & sum = \sum^{length-1}_{j = 0} array[j] \wedge \\
                    & length = length_0 \wedge \\
                    & array = array_0
                } & \\
            %
            %
            %
            & \Rightarrow \post &
        \end{flalign*}

        Everything is proved so the program is correct.
\end{document}